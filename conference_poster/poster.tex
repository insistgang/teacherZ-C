\documentclass[final]{beamer}
\usepackage[orientation=portrait,size=a0,scale=1.4]{beamerposter}
\usepackage{graphicx}
\usepackage{booktabs}
\usepackage{tikz}
\usepackage{amsmath,amssymb}
\usepackage{multicol}
\usepackage{xcolor}
\usepackage{enumitem}

\usetheme{default}
\usecolortheme{default}

% 定义颜色
\definecolor{sotonblue}{RGB}{0,82,147}
\definecolor{sotongray}{RGB}{88,89,91}
\definecolor{lightblue}{RGB}{230,242,255}
\definecolor{accentgreen}{RGB}{0,128,128}
\definecolor{accentorange}{RGB}{230,126,34}

% 设置样式
\setbeamercolor{headline}{bg=sotonblue,fg=white}
\setbeamercolor{block title}{bg=sotonblue,fg=white}
\setbeamercolor{block body}{bg=lightblue,fg=black}
\setbeamercolor{block alerted title}{bg=accentorange,fg=white}
\setbeamercolor{block alerted body}{bg=orange!10,fg=black}

\begin{document}
\begin{frame}[t]

% ==================== 标题区 ====================
\begin{beamercolorbox}[wd=\paperwidth,ht=0.1\paperheight]{headline}
\centering
\vspace{1cm}
{\Huge\bfseries From Variational Methods to Deep Learning:\\[0.5cm]
Fifteen Years of Research in Image Processing}\\[1cm]
{\LARGE Xiaohao Cai}\\[0.5cm]
{\large School of Electronics and Computer Science\\
University of Southampton}\\
\vspace{1cm}
\end{beamercolorbox}

\vspace{1cm}

% ==================== 摘要区 ====================
\begin{block}{Abstract}
\large
This poster presents fifteen years of research contributions spanning variational methods, 
deep learning, and their applications in image processing. Our work addresses fundamental 
challenges in image segmentation, radio astronomy imaging, medical image analysis, and 3D 
vision. Key innovations include the Split-and-Threshold (SaT) framework, uncertainty 
quantification methods, and efficient deep learning approaches for medical imaging.
\end{block}

\vspace{0.5cm}

% ==================== 研究时间线 ====================
\begin{block}{Research Timeline: 2011--2026}
\centering
\begin{tikzpicture}[scale=2.5]
    % 时间轴线
    \draw[line width=2pt,sotonblue] (0,0) -- (16,0);
    
    % 年份节点
    \foreach \x/\year in {0/2011, 2.5/2013, 5/2015, 7.5/2017, 10/2019, 12.5/2021, 15/2023, 16/2026} {
        \draw[line width=2pt,sotonblue] (\x,-0.2) -- (\x,0.2);
        \node[below,font=\large\bfseries] at (\x,-0.3) {\year};
    }
    
    % 里程碑
    \node[above,align=center,font=\normalsize,text width=2.5cm] at (0,0.5) {PhD Start\\Cambridge};
    \node[above,align=center,font=\normalsize,text width=2.5cm] at (2.5,1.2) {SLaT\\Framework};
    \node[above,align=center,font=\normalsize,text width=2.5cm] at (5,0.5) {Radio\\Astronomy};
    \node[above,align=center,font=\normalsize,text width=2.5cm] at (7.5,1.2) {Medical\\Imaging};
    \node[above,align=center,font=\normalsize,text width=2.5cm] at (10,0.5) {SaT\\Framework};
    \node[above,align=center,font=\normalsize,text width=2.5cm] at (12.5,1.2) {Deep\\Learning};
    \node[above,align=center,font=\normalsize,text width=2.5cm] at (15,0.5) {3D Vision\\Multimodal};
    
    % 圆点标记
    \foreach \x in {0,2.5,5,7.5,10,12.5,15} {
        \fill[accentorange] (\x,0) circle (0.15);
    }
\end{tikzpicture}
\end{block}

\vspace{0.5cm}

% ==================== 核心贡献 (4栏) ====================
\begin{columns}[t]

% ---------- 栏1: 变分图像分割 ----------
\begin{column}{0.24\textwidth}
\begin{alertblock}{\large Variational Image Segmentation}

\textbf{\large ROF Model Extensions}
\begin{itemize}[leftmargin=*]
    \item Convex relaxation techniques
    \item Multi-phase segmentation
    \item Edge-preserving regularization
\end{itemize}

\vspace{0.5cm}
\textbf{\large SLaT Framework}
\begin{itemize}[leftmargin=*]
    \item Split-Lattice-and-Threshold
    \item Efficient optimization
    \item Theoretical convergence guarantees
\end{itemize}

\vspace{0.5cm}
\textbf{\large Theoretical Contributions}
\begin{itemize}[leftmargin=*]
    \item Convergence analysis
    \item Computational complexity
    \item Generalization bounds
\end{itemize}

\vspace{0.5cm}
\begin{center}
\begin{tikzpicture}[scale=1.5]
    \node[draw,rounded corners,fill=lightblue,minimum width=3cm,minimum height=1cm] (split) at (0,2) {\textbf{Split}};
    \node[draw,rounded corners,fill=lightblue,minimum width=3cm,minimum height=1cm] (lattice) at (0,0) {\textbf{Lattice}};
    \node[draw,rounded corners,fill=lightblue,minimum width=3cm,minimum height=1cm] (thresh) at (0,-2) {\textbf{Threshold}};
    \draw[->,thick] (split) -- (lattice);
    \draw[->,thick] (lattice) -- (thresh);
\end{tikzpicture}
\end{center}

\end{alertblock}
\end{column}

% ---------- 栏2: 射电天文 ----------
\begin{column}{0.24\textwidth}
\begin{block}{\large Radio Astronomy Imaging}

\textbf{\large Uncertainty Quantification}
\begin{itemize}[leftmargin=*]
    \item Bayesian inference methods
    \item Credible intervals
    \item Error propagation analysis
\end{itemize}

\vspace{0.5cm}
\textbf{\large Online Imaging}
\begin{itemize}[leftmargin=*]
    \item Real-time processing
    \item Streaming algorithms
    \item Memory-efficient methods
\end{itemize}

\vspace{0.5cm}
\textbf{\large Deconvolution Methods}
\begin{itemize}[leftmargin=*]
    \item CLEAN algorithm variants
    \item Compressed sensing
    \item Sparse reconstruction
\end{itemize}

\vspace{0.5cm}
\textbf{\large Key Applications:}
\begin{itemize}[leftmargin=*]
    \item SKA telescope pipeline
    \item LOFAR data processing
    \item VLBI imaging
\end{itemize}

\vspace{0.5cm}
\begin{center}
\fbox{\parbox{0.8\linewidth}{\centering
\textbf{Imaging Pipeline}\\[0.3cm]
Raw Data $\rightarrow$ Calibration\\
$\rightarrow$ Gridding\\
$\rightarrow$ Deconvolution\\
$\rightarrow$ Uncertainty
}}
\end{center}

\end{block}
\end{column}

% ---------- 栏3: 医学影像 ----------
\begin{column}{0.24\textwidth}
\begin{alertblock}{\large Medical Imaging}

\textbf{\large Vessel Segmentation}
\begin{itemize}[leftmargin=*]
    \item Retinal blood vessels
    \item Coronary arteries
    \item 3D vascular structures
\end{itemize}

\vspace{0.5cm}
\textbf{\large MRI Reconstruction}
\begin{itemize}[leftmargin=*]
    \item Accelerated acquisition
    \item Compressed sensing MRI
    \item Deep learning methods
\end{itemize}

\vspace{0.5cm}
\textbf{\large Efficient Fine-tuning}
\begin{itemize}[leftmargin=*]
    \item Adapter methods
    \item LoRA techniques
    \item Parameter-efficient transfer
\end{itemize}

\vspace{0.5cm}
\textbf{\large Clinical Impact:}
\begin{itemize}[leftmargin=*]
    \item Diagnostic assistance
    \item Treatment planning
    \item Disease monitoring
\end{itemize}

\vspace{0.5cm}
\begin{center}
\begin{tikzpicture}[scale=1.2]
    \draw[thick,fill=lightblue] (0,0) rectangle (4,3);
    \node at (2,2.3) {\textbf{Medical Image}};
    \draw[thick,accentgreen] (0.5,0.5) -- (1.5,1.5) -- (2,1) -- (3,2) -- (3.5,1.5);
    \node[font=\small] at (2,0.3) {Vessel Segmentation};
\end{tikzpicture}
\end{center}

\end{alertblock}
\end{column}

% ---------- 栏4: 3D视觉与深度学习 ----------
\begin{column}{0.24\textwidth}
\begin{block}{\large 3D Vision \& Deep Learning}

\textbf{\large Point Cloud Processing}
\begin{itemize}[leftmargin=*]
    \item PointNet architectures
    \item Graph neural networks
    \item Attention mechanisms
\end{itemize}

\vspace{0.5cm}
\textbf{\large Multimodal Fusion}
\begin{itemize}[leftmargin=*]
    \item RGB-D integration
    \item LiDAR-camera fusion
    \item Cross-modal learning
\end{itemize}

\vspace{0.5cm}
\textbf{\large Deep Learning Methods}
\begin{itemize}[leftmargin=*]
    \item U-Net variants
    \item Transformer models
    \item Diffusion models
\end{itemize}

\vspace{0.5cm}
\textbf{\large Applications:}
\begin{itemize}[leftmargin=*]
    \item Autonomous driving
    \item Robotics
    \item Scene understanding
\end{itemize}

\vspace{0.5cm}
\begin{center}
\begin{tikzpicture}[scale=1.5]
    \foreach \i in {0,1,2} {
        \foreach \j in {0,1,2} {
            \fill[accentgreen] (\i*0.8,\j*0.8) circle (0.1);
        }
    }
    \draw[dashed,thick] (0,0) -- (1.6,0) -- (1.6,1.6) -- (0,1.6) -- cycle;
    \node[below,font=\small] at (0.8,-0.2) {3D Point Cloud};
\end{tikzpicture}
\end{center}

\end{block}
\end{column}

\end{columns}

\vspace{1cm}

% ==================== 方法论: SaT框架 ====================
\begin{block}{\large Methodology: Split-and-Threshold (SaT) Framework}
\centering
\begin{tikzpicture}[scale=2,
    box/.style={draw,rounded corners,fill=lightblue,minimum width=3cm,minimum height=1.5cm,align=center,font=\large},
    arrow/.style={->,thick,>=stealth}]
    
    % 输入
    \node[box,fill=sotonblue!20] (input) at (0,0) {\textbf{Input}\\Problem};
    
    % Split
    \node[box,fill=accentorange!20] (split) at (4,0) {\textbf{Split}\\Decompose into\\subproblems};
    
    % Solve
    \node[box,fill=accentgreen!20] (solve) at (8,0) {\textbf{Solve}\\Iterative\\optimization};
    
    % Threshold
    \node[box,fill=sotonblue!20] (thresh) at (12,0) {\textbf{Threshold}\\Convergence\\check};
    
    % 输出
    \node[box,fill=lightblue] (output) at (16,0) {\textbf{Output}\\Solution};
    
    % 箭头
    \draw[arrow] (input) -- (split);
    \draw[arrow] (split) -- (solve);
    \draw[arrow] (solve) -- (thresh);
    \draw[arrow] (thresh) -- (output);
    
    % 反馈循环
    \draw[arrow,dashed] (thresh.south) -- ++(0,-1) -| (solve.south) node[pos=0.25,below] {\large Iterate if not converged};
    
    % 优势
    \node[below,align=center,font=\normalsize] at (8,-2) {
        \textbf{Key Advantages:} Guaranteed Convergence $\bullet$ Computational Efficiency $\bullet$ Theoretical Foundations
    };
\end{tikzpicture}
\end{block}

\vspace{0.5cm}

% ==================== 应用案例 ====================
\begin{columns}[t]
\begin{column}{0.48\textwidth}
\begin{block}{\large Application Cases}

\begin{columns}[t]
\begin{column}{0.48\textwidth}
\textbf{1. Retinal Image Analysis}
\begin{itemize}
    \item Blood vessel detection
    \item Diabetic retinopathy screening
    \item Automated diagnosis support
\end{itemize}

\vspace{0.3cm}
\textbf{2. Radio Telescope Imaging}
\begin{itemize}
    \item SKA data processing
    \item Source detection
    \item Imaging with uncertainty
\end{itemize}
\end{column}

\begin{column}{0.48\textwidth}
\textbf{3. Medical MRI}
\begin{itemize}
    \item Fast reconstruction
    \item Artifact reduction
    \item Quantitative imaging
\end{itemize}

\vspace{0.3cm}
\textbf{4. Autonomous Driving}
\begin{itemize}
    \item 3D scene understanding
    \item Object detection
    \item Sensor fusion
\end{itemize}
\end{column}
\end{columns}

\end{block}
\end{column}

% ==================== 统计数据 ====================
\begin{column}{0.48\textwidth}
\begin{alertblock}{\large Research Statistics}

\centering
\begin{tikzpicture}[scale=1.8]
    % 统计卡片
    \node[draw,rounded corners,fill=sotonblue!20,minimum width=4cm,minimum height=2cm,align=center] (pubs) at (0,0) {
        {\Huge\bfseries 68+}\\[0.2cm]
        {\large Publications}
    };
    
    \node[draw,rounded corners,fill=accentorange!20,minimum width=4cm,minimum height=2cm,align=center] (cites) at (5,0) {
        {\Huge\bfseries 2000+}\\[0.2cm]
        {\large Citations}
    };
    
    \node[draw,rounded corners,fill=accentgreen!20,minimum width=4cm,minimum height=2cm,align=center] (collab) at (10,0) {
        {\Huge\bfseries 50+}\\[0.2cm]
        {\large Collaborators}
    };
    
    \node[draw,rounded corners,fill=lightblue,minimum width=4cm,minimum height=2cm,align=center] (students) at (2.5,-2.5) {
        {\Huge\bfseries 20+}\\[0.2cm]
        {\large PhD Students}
    };
    
    \node[draw,rounded corners,fill=lightblue,minimum width=4cm,minimum height=2cm,align=center] (grants) at (7.5,-2.5) {
        {\Huge\bfseries £5M+}\\[0.2cm]
        {\large Research Grants}
    };
\end{tikzpicture}

\end{alertblock}
\end{column}
\end{columns}

\vspace{1cm}

% ==================== 页脚 ====================
\begin{beamercolorbox}[wd=\paperwidth,ht=0.03\paperheight]{headline}
\centering
\vspace{0.5cm}
{\large \textbf{Contact:} x.cai@soton.ac.uk \quad | \quad \textbf{Website:} www.southampton.ac.uk/\textasciitilde xcai \quad | \quad \textbf{Google Scholar:} Xiaohao Cai}
\end{beamercolorbox}

\end{frame}
\end{document}
